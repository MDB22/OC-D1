As per 3.1.4. there will be two options for flight termination on the completed craft. The higher level Raspberry Pi will be able to send temrination signals to the Pixhawk directly if termination is required.  On a lower level, the Pixhawk will be able to automatically comply with termination requirements through a failsafe if necessary (such as crossing a geofence or losing signal). The Raspberry Pi will act as a last resort if the Pixhawk fails to terminate, as it relies on a seperate power supply. 

In the event that the aircraft must terminate its flight, either by the Pixhawk or Pi, controls will be completely overridden to ensure servo positions are in termination positon for fixed-wing flight, or the throttle is closed during multirotor flight (VTOL) as per 3.1.5.

As per 3.1.6. Flight termination will be automatically activated if the aircraft crosses a Geofence boundary, if the Geofence detection system fails, or if the autopilot has failed. Manual termination can also be activated by sending a signal to the Pi at the request of judges/range personnel, for example, if the aircraft appears out of control.