As per 3.1.4, there will be two options for flight termination on the completed craft. The higher level Raspberry Pi will be able to send termination signals to the Pixhawk directly if termination is required, such as Geofence crossing or loss of radio contact.  On a lower level, the PixHawk firmware will be modified to comply with termination requirements from the Raspberry Pi, or to automatically engage termination if communication with the Pi is lost.\\

In the event that the aircraft must terminate its flight, either by the PixHawk or Raspberry Pi, control will be overridden and commands will be given to each control surface as per 3.1.5, depending on the flight mode of the aircraft. If the aircraft loses communication with the PixHawk, then control of the motors and control surfaces is no longer possible; the Raspberry Pi will then disengage power to the aircraft, causing vertical descent (rotor mode) or glide (fixed-wing mode). \\

As per 3.1.6, the flight termination protocols stated above will be automatically activated if the aircraft crosses a Geofence boundary, if the Geofence detection system fails, or if the autopilot has failed or ``locked up''. Flight termination can also be activated manually by sending a signal to the Raspberry Pi at the request of judges or range safety personnel, for example, if the aircraft is deemed to be out of control.
