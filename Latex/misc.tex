\subsubsection*{Power Systems}
In order to maximise safety and reliability, the aircraft will be equipped with several independent power supplies. The sensors, servos and Raspberry Pi will be powered by the main LiPo battery system. The PixHawk and motors will be powered by a separate LiPo system, with a backup power system, so that in the event flight termination is required there will be several redundancies to ensure successful activation. This will also increase the likelihood of manual override if failures occur.

\subsubsection*{Communications}
The aircraft will make use of \href{https://www.ubnt.com/airmax/rocketm/}{Rocket M M5}  5GHz transmitters to maintain telemetry radio communications during flight, as per item 6 of the \textit{General Requirements}. This system is preliminary, and subject to investigation of the ACMA spectrum licences.

\subsubsection*{Safety Systems}
The aircraft will be equipped with an external emergency stop button, red in colour with yellow surrounding disk, to disengage power, as per item 7 of the \textit{General Requirements}. It will also be equipped with an external arming switch, and a visual state indicator to indicate armed (red) and disarmed (green) states, as per item 8 of the \textit{General Requirements}.

\subsubsection*{Storage Compartment}
The aircraft will be fitted with a storage compartment of appropriate dimensions in it's center, allowing for Joe to deposit the sample described in 1.4.1.

\subsubsection*{Mission Display}
The ground station will make use of Ardupilot \href{http://planner.ardupilot.com/}{Mission Planner} to provide a graphical display and data feed of the aircraft's mission, per 3.2.2. In addition, bespoke software will be developed to visualise the data sent back by the aircraft.
